\documentclass[]{final_report}
\usepackage{graphicx}
\graphicspath{ {./images/} }
\usepackage{hyperref}
\usepackage{amssymb}


%%%%%%%%%%%%%%%%%%%%%%
%%% Input project details
\def\studentname{Cougar Tasker}
\def\reportyear{2023-24}
\def\projecttitle{Resourceful Robots}
\def\supervisorname{Dr. Anand Subramoney}
\def\degree{MSci (Hons) in Computer Science (Artificial Intelligence)}
\def\fullOrHalfUnit{Full Unit} % indicate if you are doing the project as a Full Unit or Half Unit
\def\finalOrInterim{Interim Report} % indicate if this document is your Final Report or Interim Report

\begin{document}

\maketitle

%%%%%%%%%%%%%%%%%%%%%%
%%% Declaration

\chapter*{Declaration}

This report has been prepared on the basis of my own work. Where other published and unpublished source materials have been used, these have been acknowledged.

\vskip3em

Word Count: 

\vskip3em

Student Name: \studentname

\vskip3em

Date of Submission: 

\vskip3em

Signature:

\newpage

%%%%%%%%%%%%%%%%%%%%%%
%%% Table of Contents
\tableofcontents\pdfbookmark[0]{Table of Contents}{toc}\newpage

%%%%%%%%%%%%%%%%%%%%%%
%%% Your Abstract here

\begin{abstract}

This document serves as a layout and formatting template for your project report. It does not tell you how to write it, or what it should contain. It explains how it should be formatted and typeset. Please refer to your project booklet for information about report sizes, contents and rules.

\textbf{\textit{NOTE: in your report, you should replace this with an appropriate Abstract for your project report.}}

\end{abstract}
\newpage

%%%%%%%%%%%%%%%%%%%%%%
%%% Project Spec

\chapter*{Project Specification}
\addcontentsline{toc}{chapter}{Project Specification}
Your project specification goes here.

%%%%%%%%%%%%%%%%%%%%%%
%%% Introduction
\chapter{Introduction}

The project report is a very important part of your project and its preparation and presentation should be of extremely high quality. Remember that a significant portion of the marks for your project are awarded for this report. 

The format of the final report is fixed by the template of this document and the Department of Computer Science suggests its usage. 

While this may sound like a rather prescriptive approach to report writing, it is introduced for the following reasons:
\begin{enumerate}
 \item The template allows students to focus on the critical task of producing clear and concise content, instead of being distracted by font settings and paragraph spacing. 
 \item By providing a comprehensive template the Department benefits from a consistent and professional look to its internal project reports.
\end{enumerate}

The remainder of this document briefly outlines the main components and their usage.

A \textbf{final project report} is approximately 15,000 words and must include a word count. It is acceptable to have other material in appendixes.  
Your \textbf{interim report} for the December Review meeting, even if it is a collection of reports, should have a total word count of about 5,000 words. 
This should summarise the work you have done so far, with sections on the theory you have learnt and the code that you have written.

Also remember that any details of report content and submission rules, as well as other deliverables, are defined in the project booklet

\section{How to use this template}

The simplest way to get started with your report is to save a copy of this document. 
First change the values for the initial document definitions such as \verb|studentname| and \verb|reportyear| to match your details.
Delete the unneeded sections and start adding your own sections using the styles provided.
Before submission, remember to fill in the Declaration section fields.

\chapter{Markov Decision Processes} 

Markov Decision Processes (MDP) provide a mathematical formalisation of problems that involve decision-making. Markov Decision Processes provide the foundation that reinforcement learning (RL) is built upon. This is because MDPs distil the fundamental parts of decision-making allowing RL techniques that are built upon MDPs to generalise to learning in the real world and across different domains such as finance and robotics. 

As a formal mathematical framework, MDPs allow us to derive and prove statements about our RL methods built upon them. An important example of this is that we can prove that Q-learning (an RL technique explained in chapter~\ref{chap:q-learning}) will converge to the true Q-values as long as each Action-State pair is visited infinitely often. \cite{watkins1992q}. Furthermore, MDPs allow us to reason about problems with uncertainty allowing RL agents to account for randomness in their environment. 

The standardisation of decision-making problems as MDPs allows for a uniform definition of optimality with the value functions. this gives a basis for assessing the performance of RL algorithms. Fascinating like-for-like comparisons for different RL approaches. 


\section{Markov Property}

The Markov property is that the future state of a Markov system only depends on the current state of the system. In other words, if we have a system that follows the Markov property then the history preceding the current configuration of the system will not influence the following state, all of the information about the system is encoded in the system's current state. 

To put the Markov property formally $S_t$ represents the state at some time $t$. $S_t$ represents the outcome of some random variable. Then the Markov property would hold if and only if:

$$\Pr(S_{c+1}\ |\ S_{c},S_{c-1},\dots,S_0) = \Pr(S_{c+1}\ |\ S_{c})$$

This definition demonstrates how the Markov property can hold in non-deterministic, stochastic processes. It also shows that predictions that are only based on the current state are just as good as those that record the history in a Markov process. The sequence of events in this definition, $S_t$, is called a Markov Chain\cite{meyn2012markov}.

\newpage
\section{Extending Markov Chains}

Markov Decision Processes extend Markov Chain's in two important ways. Firstly MDPs introduce decision-making through actions. Each state in an MDP has a set of available actions in that state. In each state an action is required to transition to the next state, this action with the current state can affect what the following state will be. Secondly, MDPs introduce a reward value this reward value is produced under each state and action simultaneously with the following state.

A formal definition of a Markov Decision Process is a tuple $(\mathcal{S},\mathcal{A}_s,p)$ where:
\begin{itemize}
  \item $\mathcal{S}$ defines the set of all states
  \item $\mathcal{A}_s$ defines the set of available actions in state $s$
  \item $p$ defines the relationship between states, actions and rewards: \\
  $p(s',r\ |\ s,a) \doteq \Pr(S_{t+1} = s', R_{t+1} = r | S_t = s, A_t = a)\cite{sutton2018reinforcement}$
  \begin{itemize}
    \item $s, s' \in \mathcal{S}$, $a \in \mathcal{A}_s$ and $r \in \mathbb{R}$
    \item $p: \mathcal{S} \times \mathbb{R} \times \mathcal{S} \times \mathcal{A} \rightarrow [0,1]$
  \end{itemize}
\end{itemize}

the function $p$ is an important part of this definition it fully describes how the system will evolve, we call this function the dynamics of the MDP. what this definition does not describe is how actions are chosen. this decision-making is done by an entity called an agent. For our purposes, the agent will have full visibility as to the current state of the MDP however like most real-world situations our agent will not have any a priori knowledge of the dynamics. 

\begin{figure}[h]
  \centering
  \fboxsep 2mm
  \framebox{
    \includegraphics[width=6cm]{agent-enviroment} 
  }
  \caption{\label{fig:agent-enviroment} The agent-environment interface}
\end{figure} 
% stateDiagram-v2
%     direction LR
%     a : Agent
%     e : Enviroment
    
%     a --> e : Action
%     e --> a : Reward
%     e --> a : State

The agent comprises the entire decision-making entity in an MDP, anything that is unknown or not completely under the agent's control is referred to as the agent's environment. In the context of reinforcement learning the environment is essentially the dynamics of the MDP. Figure~\ref{fig:agent-enviroment} demonstrates how the agent and environments affect each other in an MDP. 

For learning agents we wish to improve the agent's behaviour over time for this purpose we introduce a policy $\pi$. This policy defines the action chosen by an agent under a particular state, this can be represented with a lookup table like in Q-learning\ref{chap:q-learning} or a more complex process such as with deep Q-learning. What is important about the policy is that we can update the policy based on the information the agent learns from the environment. 



\chapter{Policy and value function}
Report on notions of policy and value function; 

\section{optimal policy/value function via the Bellman equation}

\section{Finding optimal policies by iteration}

\subsection{Value iteration}
\subsection{Policy iteration}


\chapter{Q-learning}\label{chap:q-learning}

\chapter{Page Layout \& Size}

The page size and margins have been set in this document. These should not be changed or adjusted. 

In addition, page headers and footers have been included. They will be automatically filled in, so do not attempt to change their contents.

\chapter{Headings}

Your report will be structured as a collection of numbered sections at different levels of detail. For example, the heading to this section is a first-level heading and has been defined with a particular set of font and spacing characteristics. At the start of a new section, you need to select the appropriate \LaTeX{} command, \verb|\chapter| in this case.
\section{Second Level Headings}
Second level headings, like this one, are created by using the command \verb|\section|.
\subsection{Third Level Headings}
The heading for this subsection is a third level heading, which is obtained by using command \verb|\subsection|. In general, it is unlikely that fourth of fifth level headings will be required in your final report. Indeed it is more likely that if you do find yourself needing them, then your document structure is probably not ideal. So, try to stick to these three levels.
\section{A Word on Numbering}
You will notice that the main section headings in this document are all numbered in a hierarchical fashion. You don't have to worry about the numbering. It is all automatic as it has been built into the heading styles. Each time you create a new heading by selecting the appropriate style, the correct number will be assigned. 


\chapter{Presentation Issues}

\section{Figures, Charts and Tables}

Most final reports will contain a mixture of figures and charts along with the main body of text. The figure caption should appear directly after the figure as seen in Figure~\ref{fig:logo} whereas a table caption should appear directly above the table. Figures, charts and tables should always be centered horizontally. 

\begin{figure}[h]
\centering
\fboxsep 2mm
\framebox{
	\includegraphics[width=6cm]{logo} 
}
\caption{\label{fig:logo} Logo of RHUL.}
\end{figure} 

\section{Source Code}

If you wish to print a short excerpt of your source code,  ensure that you are using a fixed-width sans-serif font such as the Courier font. By using the \verb|verbatim| environment your code will be properly indented and will appear as follows:

\begin{verbatim}
static public void main(String[] args) {
  try  {
    UIManager.setLookAndFeel(UIManager.getSystemLookAndFeelClassName());
  }
  catch(Exception e) {
    e.printStackTrace();
  }
  new WelcomeApp();
} 
\end{verbatim}

\chapter{References}

Use one consistent system for citing works in the body of your report. Several such systems are in common use in textbooks and in conference and journal papers. Ensure that any works you cite are listed in the references section, and vice versa. 

\chapter{Project Information and Rules}

The details about how your project will be assessed, as well as the rules you must follow for this final project report, are detailed in the project booklet.

\textbf{You must read that document and strictly follow it.}


%%%% ADD YOUR BIBLIOGRAPHY HERE
\newpage

\bibliographystyle{IEEEtran}
\bibliography{refrences}
\addcontentsline{toc}{chapter}{Bibliography}
\label{endpage}



\end{document}

\end{article}
