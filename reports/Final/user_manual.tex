\hypertarget{running-the-program}{%
\subsection{Running the Program}\label{running-the-program}}

\begin{quote}
Note: This guide is for those only interested in using the application,
see \protect\hyperlink{development}{Development} for a development
workflow.
\end{quote}

This project requires Python 3.10. Python installers can be found
\href{https://www.python.org/downloads/}{here}. The program has a GU

\begin{quote}
If you encounter issues or don't want to install this program locally
there is a docker file \href{../.devcontainer/Dockerfile}{here}. This
docker file specifies the environment and all the tooling for this
project.
\end{quote}

You can install and run the project with a Python package manager such
as \texttt{pipx} or \texttt{poetry}. we will be using \texttt{pipx} here
because of its simplicity. There is a guide
\href{https://pypa.github.io/pipx/installation/}{here} on how you can
install \texttt{pipx}.

once \texttt{pipx} is installed open your terminal and navigate to this
folder (\texttt{code}) and run this command to start the program

\begin{Shaded}
\begin{Highlighting}[]
\ExtensionTok{python3}\NormalTok{ {-}m pipx run {-}{-}spec . start}
\end{Highlighting}
\end{Shaded}

\hypertarget{project-structure}{%
\subsection{Project Structure}\label{project-structure}}

the project is mostly hierarchical with related features being located
in the same or nearby folders (packages). the root folder \texttt{code}
contains all the meta configuration for this project, this includes -
\texttt{pyproject.toml}: which specifies what dependencies the program
needs. the entry points and other tooling configurations. -
\texttt{setup.cfg}: contains the configuration for tools that have not
migrated to the PEP 518 (\texttt{pyproject.toml}) standard yet -
\texttt{.pre-commit-config}: specifies what should be run before a
commit is allowed. - \texttt{.gitignore}: excludes certain files from
being added to the git repository.

the \texttt{code} folder contains two more important folders
\texttt{src} and \texttt{tests}. tests contains all the tests and the
related mocks necessary, code in the tests package is not used at
runtime.

\texttt{src} folder contains everything necessary at runtime, these are
in four main parts: - \texttt{model}: where all of the state and
learning functionality is stored - \texttt{view}: where all of the GUI
code for visualizing the reinforcement learning is stored -
\texttt{controller}: the code that updates the model with the user's
input, this is the code that unites the model and view. -
\texttt{entry\ points}: this is where execution starts. There are two,
the main entry point and one for profiling the code.

\texttt{src} is mostly code however it also includes the application's
config \href{./src/config.toml}{\texttt{src/config.toml}} and the icon
files used by the program. This
\href{./src/config.toml}{\texttt{src/config.toml}} is where important
configuration options for the application are determined such as the
size of the grid world. The icons are from Flaticon and require the
following attribution:

UIcons by \href{https://www.flaticon.com/uicons}{Flaticon}

\newpage
\hypertarget{documentation}{%
\subsection{Documentation}\label{documentation}}

Documentation is written alongside the code with doc-strings, in the
Google docstring format. This can be built into a complete documentation
site. a package called \texttt{mkdocs} has been configured for building
this site.

The last build of the docs is at \texttt{code/docs} you can open these
files directly or run one of the following commands to start a local
server then open \url{http://localhost:3000} in your browser to view the
documentation.

To start a local server of the last build you can run.

\begin{Shaded}
\begin{Highlighting}[]
\ExtensionTok{python3}\NormalTok{ {-}m http.server {-}{-}directory docs 3000}
\end{Highlighting}
\end{Shaded}

Alternatively use the following command to get the latest docs as a
local server.

\begin{Shaded}
\begin{Highlighting}[]
\ExtensionTok{poetry}\NormalTok{ run mkdocs serve {-}a localhost:3000}
\end{Highlighting}
\end{Shaded}

\hypertarget{development}{%
\subsection{Development}\label{development}}

To develop on this project I recommend using
\href{https://code.visualstudio.com/download}{Visual Studio Code} and
the provided
\href{https://code.visualstudio.com/docs/devcontainers/containers}{development
container}. using VS Code and this development container standardises
the environment and avoids device-specific issues and this repository is
configured for a VS Code workflow.

If not using VS Code, the required tools are: - Python 3.10, including:
- Virtual environments (\texttt{venv}) - \texttt{tkinter}/ Tcl -
\texttt{pip} - \texttt{poetry} - \texttt{pre-commit}

\hypertarget{poetry}{%
\subsubsection{Poetry}\label{poetry}}

Poetry is a Python package manager and it manages the Python retirements
of this project.

When you first start you will need to run the following command to
install the dependencies

\begin{Shaded}
\begin{Highlighting}[]
\ExtensionTok{poetry}\NormalTok{ install}
\end{Highlighting}
\end{Shaded}

you will also need to run the following command, to ensure each commit
meets the linting requirements:

\begin{Shaded}
\begin{Highlighting}[]
\ExtensionTok{poetry}\NormalTok{ run pre{-}commit install}
\end{Highlighting}
\end{Shaded}

To run the project run:

\begin{Shaded}
\begin{Highlighting}[]
\ExtensionTok{poetry}\NormalTok{ run start}
\end{Highlighting}
\end{Shaded}

\hypertarget{code-quality-tooling}{%
\subsubsection{Code quality tooling}\label{code-quality-tooling}}

To avoid bugs and enforce consistency this project has a number of
tools. these tools are configured with \texttt{pre-commit} to run
together before each commit, all tools must pass before a commit can be
pushed.

the tools are:

\begin{longtable}[]{@{}lll@{}}
\toprule
Tool & Description & command\tabularnewline
\midrule
\endhead
\texttt{black} & Formats the code in a consistent way. &
\texttt{poetry\ run\ black\ src}\tabularnewline
\texttt{pytest} & Ensures all tests are passing. &
\texttt{poetry\ run\ pytest\ -\/-cov}\tabularnewline
\texttt{flake8} & Lints the code, ensure that it meets standard. &
\texttt{poetry\ run\ flake8}\tabularnewline
\texttt{mypy} & Ensures all static types are correct. &
\texttt{poetry\ run\ mypy}\tabularnewline
\bottomrule
\end{longtable}

they can be all run together with:

\begin{Shaded}
\begin{Highlighting}[]
\ExtensionTok{pre{-}commit}\NormalTok{ run {-}{-}all{-}files}
\end{Highlighting}
\end{Shaded}

\hypertarget{tips}{%
\subsubsection{Tips}\label{tips}}

\href{https://readthedocs.org/projects/wemake-python-styleguide/}{we
make python} is the style guide that this project follows have a look at
the
\href{https://wemake-python-styleguide.readthedocs.io/en/latest/}{docs
here} for an explanation of their rules, some I have found to be
contradictory with other tooling or excessively restrictive so I have
disabled them

if the code needs formatting then in the first pass of pre-commit it
will update the code but it will fail the test because it needed to be
updated, re-run pre-commit/ try committing again and it will often
succeed
