\documentclass[]{final_report}
\usepackage{graphicx}
\usepackage{hyperref}


%%%%%%%%%%%%%%%%%%%%%%
%%% Input project details
\def\studentname{Cougar Tasker}
\def\reportyear{2023-24}
\def\projecttitle{Resourceful Robots}
\def\supervisorname{Dr. Anand Subramoney}
\def\degree{MSci (Hons) in Computer Science}
\def\fullOrHalfUnit{Full Unit} % indicate if you are doing the project as a Full Unit or Half Unit
\def\finalOrInterim{Project Plan} % indicate if this document is your Final Report or Interim Report

\begin{document}

\maketitle

%%%%%%%%%%%%%%%%%%%%%%
%%% Declaration

\chapter*{Declaration}

This report has been prepared on the basis of my own work. Where other published and unpublished source materials have been used, these have been acknowledged.

\vskip3em

Word Count: 

\vskip3em

Student Name: \studentname

\vskip3em

Date of Submission: 

\vskip3em

Signature:

\newpage

%%%%%%%%%%%%%%%%%%%%%%
%%% Table of Contents
\tableofcontents\pdfbookmark[0]{Table of Contents}{toc}\newpage

%%%%%%%%%%%%%%%%%%%%%%
%%% Your Abstract here

\begin{abstract}

  Autonomous robots such as Boston-dynamic's Spot are increasing in prevalence across various domains\cite{hagele2016robots}. Furthermore, these robotics are becoming integral to modern society, completing more advanced tasks\cite{zaouter2020autonomous}. As autonomous robots take on more complex and resource-intensive tasks, optimising their resource consumption, particularly energy, becomes paramount.
  
  these autonomous robots act in varying environments with different objectives. this project intends to explore reinfomrnet learning as a solution for prioritising these objectives, resources. 

  Reinforcement Learning is a type of machine learning where an agent learns to make decisions by interacting with an environment. specifically unlike other types of machine learning the reward are delayed after when some actions have been performed, this makes it an ideal framework for this type of problem
  
  in this project we will be simulating the environment, as a grid world or later with some more advanced environments from OpenAI's gym library
  
  

  \section*{Motivations}

  My inspiration for studying this degree in artificial intelligence comes in large part from my belief that AI is becoming increasingly pivotal in shaping the future of technology and industry. For this purpose, this project presents an invaluable opportunity for my personal and professional growth. It is a fantastic platform to improve my comprehension of reinforcement learning while offering hands-on experience. 

  What is unique about this resource-gathering robot project is its structured progression of complexity, Starting from fundamental concepts and culminating in advanced techniques. This gradient makes the complex nature of reinforcement learning more approachable than it may be in industry. 
  
  This project interests me because of its generality and applicability to many different scenarios. Resource-gathering has the potential to incorporate many real-world constraints like energy, visibility and obstacles. I would like to see how this impacts different exploration strategies.
   
  Last year, I completed my year-long internship at Zing, a digital communications company that is progressively incorporating AI systems for its customers. This experience has demonstrated to me the value of understanding the internals of these AI systems. <<insert reference to the increased value of ai in business>>. Through this project, I aim to improve my understanding of autonomous agents' benefits, biases, and limitations. This knowledge will be desirable for many companies like Zing working with artificial agents.


  \section*{Objectives}

  The primary objective is to develop and evaluate reinforcement learning agents. where the agent must navigate, make decisions, and gather resources while managing its energy reserves. the project will include two parts the report and a graphical program that implements the 
  
  This project intends to achieve the following goals:
  \begin{itemize}
    \item Energy-Efficient Navigation: Create a robot that can autonomously navigate a grid world while making optimal decisions to conserve energy. The robot should learn to prioritise energy-efficient paths.
    \item Resource Gathering Strategies: Design intelligent resource-gathering strategies for the robot, ensuring it collects essential resources while balancing energy expenditure. The robot should adapt its behaviour based on the availability of resources and its current energy levels.
    \item Dynamic Energy Management: The robot should monitor its energy reserves and adjust its actions accordingly. This includes learning when to engage in resource gathering and when to return to a charging station.
    \item Effective Exploration: Develop exploration strategies that enable the robot to learn and adapt to different grid world scenarios.
  \end{itemize}
  
  Extension:
  
  an extension would be to implement deep reinforcement learning, this is integrating artificial neural networks with Q-learning. this unlocks working in environments with far larger observations and action spaces. this can be used to extend our existing environment or a further extension of integrating a different environment such as one from OpenAI's gym library.

\end{abstract}
\newpage

%%%%%%%%%%%%%%%%%%%%%%
%%% Project Spec

\chapter*{Project Specification}
\addcontentsline{toc}{chapter}{Project Specification}
Your project specification goes here.

%%%%%%%%%%%%%%%%%%%%%%
%%% Introduction
\chapter{Introduction}

The project report is a very important part of your project and its preparation and presentation should be of extremely high quality. Remember that a significant portion of the marks for your project are awarded for this report. 

The format of the final report is fixed by the template of this document and the Department of Computer Science suggests its usage. 

While this may sound like a rather prescriptive approach to report writing, it is introduced for the following reasons:
\begin{enumerate}
 \item The template allows students to focus on the critical task of producing clear and concise content, instead of being distracted by font settings and paragraph spacing. 
 \item By providing a comprehensive template the Department benefits from a consistent and professional look to its internal project reports.
\end{enumerate}

The remainder of this document briefly outlines the main components and their usage.

A \textbf{final project report} is approximately 15,000 words and must include a word count. It is acceptable to have other material in appendixes.  
Your \textbf{interim report} for the December Review meeting, even if it is a collection of reports, should have a total word count of about 5,000 words. 
This should summarise the work you have done so far, with sections on the theory you have learnt and the code that you have written.

Also remember that any details of report content and submission rules, as well as other deliverables, are defined in the project booklet~.

\section{How to use this template}
\subsection{Test}
\subsubsection{Test}
The simplest way to get started with your report is to save a copy of this document. 
First change the values for the initial document definitions such as \verb|studentname| and \verb|reportyear| to match your details.
Delete the unneeded sections and start adding your own sections using the styles provided.
Before submission, remember to fill in the Declaration section fields.

\chapter{Page Layout \& Size}

The page size and margins have been set in this document. These should not be changed or adjusted. 

In addition, page headers and footers have been included. They will be automatically filled in, so do not attempt to change their contents.

\chapter{Headings}

Your report will be structured as a collection of numbered sections at different levels of detail. For example, the heading to this section is a first-level heading and has been defined with a particular set of font and spacing characteristics. At the start of a new section, you need to select the appropriate \LaTeX{} command, \verb|\chapter| in this case.
\section{Second Level Headings}
Second level headings, like this one, are created by using the command \verb|\section|.
\subsection{Third Level Headings}
The heading for this subsection is a third level heading, which is obtained by using command \verb|\subsection|. In general, it is unlikely that fourth of fifth level headings will be required in your final report. Indeed it is more likely that if you do find yourself needing them, then your document structure is probably not ideal. So, try to stick to these three levels.
\section{A Word on Numbering}
You will notice that the main section headings in this document are all numbered in a hierarchical fashion. You don't have to worry about the numbering. It is all automatic as it has been built into the heading styles. Each time you create a new heading by selecting the appropriate style, the correct number will be assigned. 


\chapter{Presentation Issues}

\section{Figures, Charts and Tables}

Most final reports will contain a mixture of figures and charts along with the main body of text. The figure caption should appear directly after the figure as seen in Figure~\ref{fig:logo} whereas a table caption should appear directly above the table. Figures, charts and tables should always be centered horizontally. 

\begin{figure}[h]
\centering
\fboxsep 2mm
\framebox{
	\includegraphics[width=6cm]{logo} 
}
\caption{\label{fig:logo} Logo of RHUL.}
\end{figure} 

\section{Source Code}

If you wish to print a short excerpt of your source code,  ensure that you are using a fixed-width sans-serif font such as the Courier font. By using the \verb|verbatim| environment your code will be properly indented and will appear as follows:

\begin{verbatim}
static public void main(String[] args) {
  try  {
    UIManager.setLookAndFeel(UIManager.getSystemLookAndFeelClassName());
  }
  catch(Exception e) {
    e.printStackTrace();
  }
  new WelcomeApp();
} 
\end{verbatim}

\chapter{References}

Use one consistent system for citing works in the body of your report. Several such systems are in common use in textbooks and in conference and journal papers. Ensure that any works you cite are listed in the references section, and vice versa. 

\chapter{Project Information and Rules}

The details about how your project will be assessed, as well as the rules you must follow for this final project report, are detailed in the project booklet~.

\textbf{You must read that document and strictly follow it.}


%%%% ADD YOUR BIBLIOGRAPHY HERE
\newpage

\bibliographystyle{plain}
\bibliography{refrences}
\addcontentsline{toc}{chapter}{Bibliography}
\label{endpage}



\end{document}

\end{article}
